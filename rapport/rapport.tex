\documentclass{article}

\author{Sebastien Patte \\ Naïm Moussaoui-Remil}
% \usepackage[margin=60pt]{geometry}
\usepackage{amsmath}
\usepackage{amssymb}
\usepackage{tikz}
\usepackage{xcolor}
\usepackage{caption}
\usepackage{subcaption}

\usetikzlibrary{petri,positioning}

\title{Méthodes formelles - Vérification probabiliste}

\newcommand{\sv}{\;\vert\;}
\newcommand{\wrt}{with respect to\xspace}
\newcommand{\eva}{Eva\xspace}
\renewcommand{\wp}{WP\xspace}
\newcommand{\eacsl}{E-ACSL\xspace}


\begin{document}

%\selectlanguage{french}
\maketitle
%\nocite{*}


	
\section{Modélisation}

Le modèle est composé de 2 modules : Utilisateur et Contrôleur. 
L'utilisateur peut faire les actions \texttt{ajoutPièce}, \texttt{prendGobelet},
\texttt{B}$_\texttt{ann}$, \texttt{B}$_\texttt{café}$, \texttt{B}$_\texttt{thé}$ et \texttt{B}$_\texttt{sucre}$, mais uniquement quand ces actions actions sont simultanément possibles dans le modèle du Contrôleur.

\subsection{Utilisateur}

Dans le modèle de l'utilisateur, on s'occupe pricipalement de la gestion du niveau de sucre avec le bouton \texttt{B}$_\texttt{sucre}$, les autres actions étant gérées dans le module Contrôleur. 

Les actions \texttt{ajoutPièce}, \texttt{prendGobelet},
\texttt{B}$_\texttt{ann}$, \texttt{B}$_\texttt{café}$, \texttt{B}$_\texttt{thé}$ bouclent sur l'état initial. 
Mais l'action \texttt{B}$_\texttt{sucre}$ fait passer le système dans un état où la proposition \texttt{B}$_\texttt{sucre}$ est vraie, et depuis lequel on peut seulement revenir à l'état initial par une des 2 actions internes en fonction du niveau de sucre actuel.

Le niveau de sucre : 0 (pas sucré), 1 (sucré), et 2 (très sucré), est incréménté si le niveau actuel est $<2$, et est réinitialisé à $0$ si il était à $2$. 

\usetikzlibrary {automata,positioning}
\begin{figure}[h]
	\centering
\begin{tikzpicture}[node distance=4cm,auto]
  \node[state]  (q)   {\empty};
  \node[state]  (q1) [right = of q]  {Bsucre};

  \path[->, every loop/.style={in=115,out=145}] (q) edge [loop] node {Bcafé} (); 
  \path[->] (q) edge [loop above] node        {Bthé} ();
  \path[->, every loop/.style={in=210,out=245}] (q) edge [loop] node[left] {prendGobelet} ();
  \path[->] (q)  edge node[below]  {Bsucre} (q1);
  \path[->] (q1) edge[bend left]  node[below] {sucre$<$2 ? sucre+=1} (q);
  \path[->] (q1) edge[bend right] node[above] {sucre=2 ? sucre=0} (q);
  \path[->] (q)  edge[loop below] node[below left]  {ajoutPièce} ();
  \path[->] (q)  edge[loop left]  node        {Bann} ();
\end{tikzpicture}
%\caption{FD Typestate} \label{fig:M1}
\end{figure}





\subsection{Contrôleur}

La machine a café peut être dans 4 modes différents : $\texttt{M}_\texttt{dispo}$ $\texttt{M}_\texttt{choix}$, $\texttt{M}_\texttt{prépa}$ et $\texttt{M}_\texttt{fin}$, représentés respectivement par les couleurs \textcolor{red}{rouge}, \textcolor{green}{vert}, \textcolor{blue!50}{bleu}, et \textcolor{orange!80}{orange}.

\subsubsection{Fonctionnement global}

Depuis l'état initial, quand l'Utilisateur ajoute une pièce on passe dans le mode $\texttt{M}_\texttt{choix}$. Quand la machine est dans ce mode, l'utilisateur peut : règler le niveau de sucre avec l'action $\texttt{B}_\texttt{sucre}$, appuyer sur $\texttt{B}_\texttt{ann}$ pour retourner dans $\texttt{M}_\texttt{dispo}$, ou bien sur $\texttt{B}_\texttt{café}$ ou $\texttt{B}_\texttt{thé}$ pour aller dans le mode $\texttt{M}_\texttt{prépa}$. 

La machine passe du mode $\texttt{M}_\texttt{prépa}$ à $\texttt{M}_\texttt{fin}$ quand elle a terminé de servir la boisson, puis l'utilisateur peut alors prendre le gobelet pour faire retourner la machine dans son état initial.

\begin{figure}[h]
	\centering
\begin{tikzpicture}[auto, node distance=0.8cm]
  \node[state,initial above]  (q0) [fill=red]  {Mdispo};
  \node[state]  (q1) [below=of q0, fill=green]  {Mchoix};
  \node[state]  (q2) [below=of q1, fill=blue!50]  {Mprépa};
  \node[state]  (q3) [below=of q2, fill=orange!80]  {Mfin};
  
  \path[->] 
    (q0) edge[bend right] node[left] {ajoutPièce} (q1)
    (q1) edge[bend right] node[right] {Bann} (q0)
    (q1) edge[loop right] node {Bsucre} (q1)
    (q1) edge[bend right] node[left] {Bthé} (q2)
    (q1) edge[bend left] node[right] {Bcafé} (q2)
    (q2) edge node[right] {} (q3)
    (q3) edge[bend left= 100,min distance=3cm] node[left] {Gout} (q0)
  ;
\end{tikzpicture}
\end{figure}

\subsubsection{États intermédiaires}

Pour pouvoir utiliser les propositions atomiques $\texttt{B}_\texttt{ann}$, $\texttt{B}_\texttt{café}$, $\texttt{B}_\texttt{thé}$, $\texttt{P}_{50}$ $\texttt{R}_{50}$, et $\texttt{G}_\texttt{out}$, on a en fait un état intermédiaire pour chaque action de l'utilisateur. 

Une action ($\texttt{Bann}$, $\texttt{Bcafé}$, $\texttt{Bthé}$, \texttt{ajoutePièce}, \texttt{prendGobelet}), fait atterir le système dans un état ou la proposition correspondante ($\texttt{B}_\texttt{ann}$, $\texttt{B}_\texttt{café}$, $\texttt{B}_\texttt{thé}$, $\texttt{P}_{50}$, $\texttt{G}_\texttt{out}$) est vraie, puis cette proposition passe à faux dans l'état suivant.  
De plus, depuis l'état où $\texttt{B}_\texttt{ann}$ est vraie, une action interne amène le système dans le mode $\texttt{M}_\texttt{dispo}$ avec la proposition $\texttt{R}_{50}$ vraie, puis directement dans l'état initial.

\begin{figure}[h!]
	\centering
\begin{tikzpicture}[auto, node distance=1cm, text width = 1.5cm, align=center]
  \node[state, initial]  (dispo1)[fill=red]   {Mdispo, $\lnot$p50, $\lnot$r50};
  \node[state]  (dispo2) [below=of dispo1, fill=red]  {Mdispo, p50};
  \node[state]  (dispo3) [left=of dispo2, fill=red]  {Mdispo, r50};
  
  
  \node[state]  (choix1) [below =of dispo2, fill=green]  {Mchoix, $\lnot$Ban};
  \node[state]  (choix2) [left =of choix1, fill=green]  {Mchoix, Ban};
  \node[state]  (prepCafe) [below =of choix1, fill=blue!50]  {Mprépa, Bcafé};
  \node[state]  (prepThe) [right =of prepCafe, fill=blue!50]  {Mprépa, Bthé};

  \node[state]  (finCafe) [below =of prepCafe, fill=orange!80]  {Mfin, Pcafé};
  \node[state]  (finThe) [below =of prepThe, fill=orange!80]  {Mfin, Pthé};
  \node[state]  (finGout) [below=of finCafe, fill=orange!80]  {Mfin, Gout};
  
  \path[->] 
    (dispo1) edge node[right, text width = 2cm] {ajoutPièce} (dispo2)
    (dispo2) edge node[left] {} (choix1)
    (choix1) edge[loop right] node {Bsucre} ()
    (choix1) edge node[above] {Bann} (choix2)
    (choix2) edge node {} (dispo3)
    (dispo3) edge node {} (dispo1)
    (choix1) edge node[left] {Bcafé} (prepCafe)
    (choix1) edge node {Bthé} (prepThe)

    (prepCafe) edge node {} (finCafe)
    (prepThe) edge node {} (finThe)
    (finCafe) edge node[left,text width = 2.2cm] {prendGobelet} (finGout)
    (finThe) edge node[text width = 2.2cm, below] {prendGobelet} (finGout)
    (finGout) edge[bend right=100] node {} (dispo1)
  ;  
\end{tikzpicture}
\end{figure}

\clearpage

\section{Spécification}
Nous allons reprendre quelques propriétés où une petite explication est nécessaire.
\begin{enumerate}
    \item Ici la seul difficulté et d'exprimer un "xor" pour chaque cas d'où la présence de mutiple négation.
\begin{verbatim}
Prop: A [ G (("Mdispo"&!("Mfin")&!("Mprepa")&!("Mchoix")) ....]
\end{verbatim}
    \item Dans le fichier \texttt{.props}, cette proporiété est séparée en 3. 
        Rien de spécial à préciser. 
    \item Pour indiquer le fait que le changement d'état arrivera un temps plus tard en utilisant l'implication et l'opérateur next.
\begin{verbatim}
Prop :A [ G ("Mdispo"&"P50" $\longrightarrow$ (X ("Mchoix"))) ].
\end{verbatim}
    \item Rien de particulier. Le "mais si" est traité comme une disjonction donc on obtient deux sous propriétés.
    \item RAS.
    \item Dans la première propriété, il est précisé que l'évenement arriverra "éventuellement". Ce qui nous indique qu'il faut utiliser l'opérateur F.
\begin{verbatim}
Exemple prop 1: A [ G ("Mdispo"&(X (!"Mdispo"))=>(X(F"Mchoix"))) ].
\end{verbatim}
    Dans les 3 autres, c'est directement à l'état d'après donc on utilise un next uniquement.
    \item Le défi est de traduire "infiniment souvent". Ce que l'on traduit par "toujours un jours" soit "G F". 
    \item Dans le fichier \texttt{.props}, cette proporiété est séparée en 3. 
    Rien de spécial à préciser. 
    \item En appuyant sur le bouton café on aura un jour un café. Donc utilise l'opérateur F encore. 
\end{enumerate}
\end{document}